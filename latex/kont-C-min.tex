

\documentclass{article}
\usepackage[utf8]{inputenc}
\usepackage{setspace}
\usepackage{ mathrsfs }
\usepackage{amssymb} %maths
\usepackage{amsmath} %maths
\usepackage[margin=0.2in]{geometry}
\usepackage{graphicx}
\usepackage{ulem}
\setlength{\parindent}{0pt}
\setlength{\parskip}{10pt}
\usepackage{hyperref}
\usepackage[autostyle]{csquotes}

\usepackage{cancel}
\renewcommand{\i}{\textit}
\renewcommand{\b}{\textbf}
\newcommand{\q}{\enquote}
%\vskip1.0in



\begin{document}

\begin{huge}

{\setstretch{0.0}{
Kontinuum [ C Version ]

This program displays the evolution of a continuous cellular automaton which is itself randomly evolving. What appears as a rectangular grid is computationally a torus, featuring both vertical and horizontal wraparound. This version is the leanest and maybe the fastest. I plan to add what the Rust and Go versions both have, which is a way to store filters (\q{cartridges}) for reuse. Sometimes a fun pattern appears, and it's nice to be able to generate it on demand. 

The instructions for use are easy to find in the file \b{respond.c}. I do use the Raylib library, so you may need to install that to use the program. 



}}
\end{huge}
\end{document}
